\documentclass[11pt]{article}

\usepackage[margin=1.2in]{geometry}
%\usepackage{supertabular}
%\usepackage{setspace}
%\doublespacing
%\usepackage{graphicx}
\usepackage{hyperref}

\newcommand{\company}[1]{\verb|#1|}
\newcommand{\software}[1]{\verb|#1|}
\newcommand{\project}[1]{\texttt{#1}}
\newcommand{\comment}[1]{ [ \textit{#1} ] }

\newcommand{\DD}{\mathcal{D}}
\newcommand{\WW}{\mathcal{W}}
\newcommand{\TT}{\mathcal{T}}

\author{Ivan Savov}
\title{ {\Huge Latent Dirichlet Allocation for \\ scientific topic extraction } }

\begin{document}
\maketitle

\abstract{
    We learn a topic model from a large collection of scientific articles using Latent Dirichlet Allocation.
    The data set is the entire pdf contents of the Cornell Physics pre-print archive arXiv.org for the \texttt{quant-ph} category.
    We represent each document as an array of word counts from a fixed dictionary of size $W=?$,
    which we determined by 

    We display the discovered topics and incorporate them into a rudimentary recommendation system.
}


{\bf keywords: } LDA, topic model, collapsed variational bayes, domain knowledge, 



\section{Introduction}

	When I first learned about Latent Dirichlet Allocation I was totally amazed by the seemingly magical ability
	of this algorithm to automatically discern the topics underlying a collection of documents.
	Since then I have been thinking about how I can apply this method to  


    Terms to discuss MAP, PLSA, VB, CVB0 etc...


\section{The model}

    LDA 


\section{The algorithm}

    First we being by definiting some quantities
    
    We have the document set $\DD = \{ d_1, d_2, \ldots, d_D \}$, where each document consists of 
    a word count vector for the words taken from a fixed vocabulary $\WW$ of size $W$.
    
    Our aim is to produce a set of topics $\TT$ where each topic is a probability distribution over words $\WW$.
    
    
    \begin{description}
	\item[$\gamma_{wjk}$:]	$=Pr( z=k | x=w,d=j)$, the probability that word $w$ in document $j$ 
						is assigned to the topic k.
	\item[$N_{wj}$:]	\# of times $w$ appears in doc $j$. 
	\item[$N_{wk}$:]  $=\displaystyle\sum_{j\in \DD} N_{wj}\gamma_{wjk} = $ \# of times $w$ appears in topic $k$.
	\item[$N_{kj}$:]  $=\displaystyle\sum_{w\in \WW} N_{wj}\gamma_{wjk} = $ \# of times topic $k$ appears in doc $j$.
	\item[$N_{k}$:]  $=\displaystyle\sum_{w\in \WW} N_{wk}=  \displaystyle\sum_{w\in \WW}\sum_{j\in \DD} N_{wj}\gamma_{wjk} $ \# words assigned to  topic $k$.
	\item[$N_{j}$: ]  $=\displaystyle\sum_{k \in \TT} N_{kj} =  \displaystyle\sum_{k \in \TT}\sum_{w\in \WW}N_{wj}\gamma_{wjk} $ \# words in doc $j$.
    \end{description}


    To achieve this we will have to learn...


\section{The data set}

	The good people who run the arXiv.org were kind enough to send me the entire collection of papers
	in the \texttt{quant-ph} repository. The papers contain nearly all the research in the physical sciences
	that is not particle physics.
	 
	\begin{verbatim}
	ivan@flicker:/scratch/arxiv$ du -sh *
	7.1G    pdf
	2.1G    tex
	ivan@flicker:/scratch/arxiv$ find pdf/ -type f | wc
	31739   31739  730019
	ivan@flicker:/scratch/arxiv$ find tex/ -type f | wc
	21061   21061  470945
	\end{verbatim}
	
	The \texttt{pdf/} directory contains the pdf versions of the papers and for some of these papers.
	For about $2/3$ of these we also have the latex source code in the \texttt{tex/} directory.
	
	Some more information about the dataset:
	\begin{itemize}
		\item Total number of documents: 31739
		\item Earliest date: 22 Dec 1994  (pdf/9412/9412002v1.pdf)
		\item Most recent:  30 Mar 2007 (pdf/0703/0703278v1.pdf)
	\end{itemize}
	
	
	\subsection{Intended pre-processing}
	
		I plan to use the command \texttt{pdftotext} to convert the pdf documents to text files and
		then use the standard bag-of-words paradigm for each document.
		
		I will do a first pass on the documents and remove stopwords.		
		Given how large of a dataset is available I will try to run my algorithm without stemming in order
		to capture as much of the granularity of the data.
		If this proves to be ineffective then I will use stemming.
		  


	
\section{Research project}

	I will read the original 2003 paper by Blei, Ng and Jordan \cite{Blei2003} and also perform a general literature
	review on topic models.
	In particular, I want to verify the results of the ICML09 paper \cite{Teh2009} which state that all the different 
	models of training
	LDA are equivalent and only differ by the setting of the hyper-parameters.
	
	I will try to implement the Collapsed Variational Bayes technique of \cite{Teh2007}.
	
	The results I would like to obtain is some meaningful cluster structure over the scientific topics.
	
	
\section{Next steps}
	
    Prepare python script for generating a topicmodel run. \\
    folder + symlinks\\
    docs.txt \\
    run it and time it \\ 
    check memory usage as it runs \\


    See how to classify new documents.


    Compare for different NITER and Ntopics


    Calculate perplexity and KL distance for different runs with same parameters


    
\section{Results}

	
	
	
	
\section{Discussion and future work}

	\subsection{Regular expressions}
	
		This could be a study of what you can do with different features.
		We can have as higher level features counts from regular expression matches.
		These features can themselves be refined over generations or by cross validation.
		
		Come to think of it you can do all kinds of things with this.
		Regular expressions of the form "quantum (mechanics|physics)" for compound words 
		that are synonymous. 
		
		Algorithm: Self-refining regular expression features
		\begin{enumerate}
			\item	Start with word counts for a fixed list of words of size $|W|$ and do regular LDA.
			\item	For each of the latent topics build all "two word" combinations.
			\item	Rerun LDA on the expanded feature space of size $|W|^2 + |W|$.
			\item	Find the important second order occurrences and discard all other features.
			\item	Try to compress multiple variations using regular expression language. \\
					Example: "a b", "a c", "a d"  can be converted into regex "a (b|c|d)"
			\item	Go to step 3 but now number of features is $|W|+$ n\_of\_good\_reg\_exes
		\end{enumerate}

		I am not sure if this will generate all possible word occurences but I think reg-exes are
		pretty rich as feature set and their computation is all paralelizable so it doesn't matter if 
		it is very expensive.
		
		
		Can we let some genetic algorithm work on this instead?



    \subsection{Two data set correlations}

        You browse wikipedia about quantum informaiton topics and it suggests
        papers that are most similar to the subject --- not so useful.

        Maybe the opposite is better, as you browse the arXiv you get suggested
        links to articles on wikipedia that cover similar topics.




	
	\subsection{This paper covers}
	
		User interface idea: papers laid out on a table in such a way that the papers that
		cite each other overlap. All papers in the same "epoch" are visible at the same time
		and placement and paper size indicates relative importance.
		
		(One way to think about this is where are the words of this each topic passed on from one generation
		to the next)
		 
		
	\subsection{CUDA }
			
    \subsection{Citaiton graph}
        We can use a simple reg-ex to match things like \\
        \verb| quant-?ph\d{7} | and extract all citations to other
        papers on the archive.
        It won't be very complete, but you can still extract a sense 
        of importance from it !
        page rank ;)

    \section{Metadata}
        We should have the titles for each of these articles I think
        we need a little urllib2 script to get each of
        \verb|http://arxiv.org/list/quant-ph/0003?show=1000|
        where 0003 corresponds to my folder 0003 


\section{Conclusion}

    We have demonstrated that     

\bibliographystyle{alpha}
\bibliography{arXivLDA}


\end{document}
